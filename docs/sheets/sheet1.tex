\documentclass[a4paper]{article}
\usepackage[12pt]{extsizes}
% \usepackage[14pt]{extsizes}
\usepackage[utf8]{inputenc}
\usepackage[T2A]{fontenc}
\usepackage[russian]{babel}

\usepackage{amsmath,amsthm,amsfonts,amssymb,amscd}
\usepackage{geometry}
\usepackage{fullpage}
% \usepackage{lastpage}
\usepackage{enumerate}
\usepackage{fancyhdr}
\usepackage{mathrsfs}
\usepackage{xcolor}
\usepackage{graphicx}
\usepackage{hyperref}
\usepackage{mathtools}

\setlength{\parindent}{0.0in}
\setlength{\parskip}{0.05in}


\def \incl 	{\xhookrightarrow}

\pagestyle{fancy}
\headheight 35pt
\lhead{\small{Кафедра МОУ, Общая и алгебраическая топология}}
\rhead{\small{листок 1, \today}}
% \lfoot{}
% \cfoot{}
% \rfoot{\small\thepage}
\headsep 1.5em
\pagenumbering{gobble}

\newgeometry{
    left=1.5em,
    right=2em,
    top=1.7cm,
    bottom=1.0cm,
}
\fancyfootoffset{0pt}

\begin{document}

% \setlength{\headheight}{23pt}
% \setlength{\parindent}{0.0in}
% \setlength{\parskip}{0.0in}

% \pagestyle{empty}
% 	Кафедра МОУ, Основы общей и алгебраической топологии
% }
% \flushleft{
% 	Кафедра МОУ, Основы общей и алгебраической топологии, листок 1, 1.9.2019 г.
% }

\begin{center}
	\large{\textbf{Начала общей топологии}}
\end{center}

\begin{enumerate}
\item Открытые множества в смысле окрестностной топологии --- это в точности открытые множества.
\item Подмножество замкнуто тогда и только тогда, когда содержит все свои точки прикосновения.
\item В хаусдорфовом пространстве точка --- замкнутое множество.
\item Непрерывный образ компакта компактен.
\item Замкнутое подмножество компакта компактно.
\item Если $X$ компактно, $Y$ хаусдорфово, а $f\colon X\to Y$ непрерывно, то прообраз $f^{-1}(V)$ всякого компактного подмножества $V\subset Y$ компактен в $X$.
\item Приведите пример отображения $f\colon X\times Y\to Z$, непрерывного по каждой переменной, но не непрерывного.
\item Существует ли биекция между компактными пространствами, не являющаяся гомеоморфизмом?
\item Канторово множество гомеоморфно счётному произведению множества $\{0,1\}$ на себя: $\mathcal{C}\cong\displaystyle\prod\limits_{i=1}^{\infty} \{0,1\}$.
\item Индуцированная топология на $A\subset X$ является самой грубой из всех топологий, для которых отображение $A\incl{} X$ непрерывно.
\item Фактор-топология на $X/\sim$ является самой тонкой из всех топологий, для которых отображение проекции $X\to X/\sim$ непрерывно.
\item Пространство $[0,1]/\{0,1\}$ гомеоморфно $S^1$. Сформулируйте и докажите более общее утверждение.
\item Компакт в хаусдорфовом пространстве замкнут.
\item 
% Канторово множество строится следующим образом. Возьмём отрезок $I=[0,1]$. Удалим из него интервал $(1/3,2/3)$, получим два несвязных отрезка $[0,1/3]$ и $[2/3,1]$. Теперь из них удалим срединные трети $(1/9,2/9)$ и $(7/9,8/9)$, получим четыре несвязных отрезка. Будем бесконечно повторять процедуру. Канторово множество $\mathcal{C}$ --- это пересечение всех множеств, полученных на каждом шаге процедуры. Покажите, что $\mathcal{C}\cong\displaystyle\prod\limits_{i=1}^{\infty} \{0,1\}$.
\item \textbf{(Лемма о склейке.)} Пусть пространство $X$ представлено конечным объединением замкнутых множеств $X_i$, $i=\overline{1,n}$, и заданы отображения $f_i\colon X_i\to Y$, причём если $X_{ij}=X_i\cap X_j$ непусто, то $f_i|_{X_{ij}}=f_j|_{X_{ij}}$. Тогда существует единственное непрерывное отображение $f\colon X\to Y$ такое, что $f|_{X_i}=f_i$.

\end{enumerate}
\end{document}