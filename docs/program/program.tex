\documentclass[12pt]{article}
\usepackage[left=30mm,right=30mm, top=20mm,bottom=20mm,bindingoffset=0cm]{geometry}
\usepackage[utf8]{inputenc}
\usepackage[T2A]{fontenc}
\usepackage[russian]{babel}

\relpenalty=10000
\binoppenalty=10000

\begin{document}
\pagestyle{empty}
\large{
\begin{center}
	\textbf{ОСНОВЫ ОБЩЕЙ И АЛГЕБРАИЧЕСКОЙ ТОПОЛОГИИ}\\
	Кафедра математических основ управления\\
	И. Цюцюрупа\\
	ПРОГРАММА КУРСА
\end{center}
}
\begin{enumerate}
	\item Определение топологического пространства, непрерывность в точке и вообще, примеры топологических пространств, топология на пространстве отображений. Индуцированная топология, фактортопология, тихоновская топология. Действие группы на топологическом пространстве. 
	\item Введение в язык теории категорий: категории, функторы, универсальные объекты. Применение теории категорий в алгебраической топологии.
	\item Операции над топологическими пространствами, склейка. Универсальные свойства.
	\item Гомотопия и гомотопическая эквивалентность, связь с интегралом функции комплексного переменного вдоль кривой.
	\item Клеточные пространства и теорема о клеточной аппроксимации.
	\item Фундаментальная группа топологического пространства, примеры вычисления и классические приложения: теорема Брауэра о неподвижной точке, теорема Борсука-Улама, основная теорема алгебры.
	\item Свободное произведение групп и теорема ван Кампена. Фундаментальная группа клеточного пространства. Классификация двумерных поверхностей.
	\item Накрытия, свойство поднятия пути и гомотопии. Универсальное накрытие, классификация накрытий. Теорема Нильсена-Шраера.
\end{enumerate}
Если позволит время:
\begin{enumerate}\setcounter{enumi}{8}
	\item Расслоения: локально тривиальные, в смысле Гуревича, в смысле Серра. Корасслоения.
	\item Гомотопические группы, их коммутативность. Точная последовательность пары и теорема Уайтхеда.
	\item Группы гомологий, гомологии многообразий и теорема Пуанкаре.
\end{enumerate}

\end{document}